\documentclass{report}

\begin{document}

\title{October 5, 2019\\
Software Engineering\\
CEN 4010 – U01\\
Deliverable 2\\
Blue Jay}

\author{Team 3\\
\\•	Mareo Yapp
\\•	Brandon Lee
\\•	Josselyn Ruiz
\\•	Alex Collantes
\\•	David Ramirez
\\Professor Peter Clarke}
\maketitle

\begin{abstract}

\end{abstract}
 
\tableofcontents

\chapter{Introduction}
	This chapter will introduce the purpose of BlueJay (BJ), the functional and non-functional requirements of the use cases implemented. The software process model used will be identified along with the UML models that will be shown in the document.
\section{Purpose of system}
	Social media is a constant part of everyday life. People use it to stay in touch with their friends and family and to display their interests to the world, but far too often social media companies have been lapsing in protecting their users’ information.\\
	BlueJay (BJ) allows users to post text and images in a safe and secure manner to reduce the leaking of sensitive information, which can lead to fraud and identity theft. This will be done using many existing security techniques, such as password encryption, SSL encryption, and other safety measures that encourage users to prioritize securing their data.
\section{Requirements}
	\subsection{Functional requirements}
	1.	The system will allow users to create an account (see use case BJ015 – Account Creation in Appendix A).\\
	2.	The system will allow users to create and upload posts (see use case BJ004 – Create Post in Appendix A).\\
	3.	The will allow users to reply to a post (see use case BJ006 – Reply to Post in Appendix A).\\
	4.	The system will allow users to view posts on the site (see use case BJ002 – View Post in Appendix A).\\
	5.	The system will allow users to browse most recent popular post (see use case BJ001 – Browse Post in Appendix A)\\
	6.	The system will make users go through a process to access their account (see use case BJ021 – Login in Appendix A).\\
	7.	The system will allow users to exit their accounts (see use case BJ022 – Logout in Appendix A).\\
	8.	The system will allow users to go through a process to access their account if they forget their password (see use case BJS005 – Forgot Password in Appendix A).\\
	9.	The system will make users go through a process to get their password by answering 3 security questions (see use case BJS006 – Security Questions in Appendix A).\\
	10.	The system will send users an email asking them to verify their account (see use case BJS007 – Email Verification in Appendix A).\\
	\subsection{Non-functional requirements}
	Usability: \\
	Reliability: \\
	Performance: \\
	Supportability: \\
	Implementation: \\
\section{Design methodology used} 
	The Unified Software Development Process model was used. The package, use case, deployment, ER, class, sequence and state machine diagram were used to represent the design.
\section{Definitions, acronyms, and abbreviations}
BlueJay\\StarUML\\SM User\\MVC\\Client/Server\\Three Tier
\section{Overview of document}
\chapter{Proposed Software Architecture}
	This chapter will introduce 
\section{Overview}
	MVC was used because the client needs to modify its view; so, it requests the controller which generates a new model or a portion of a model it can send back over to view.\\
	Client/Server was used
\section{Subsystem Decomposition}
\section{Hardware and Software Mapping}
\section{Persistent Data Management}
\section{Security Management}
	BlueJay prides itself on its security, to ensure this various security protocols were implemented to protect unauthorized access to SM user’s accounts. These protocols include email verification, security questions and password recovery. 
\chapter{Detailed Design}
	This chapter will introduce 
\section{Overview}
\section{State Machine}
\section{Object Interaction}
\section{Detailed Class Design}
\chapter{Glossary}
BlueJay: BlueJay (BJ) is a social media platform that allows users to post texts and images in a safe and secure environment.\\
StarUML: StarUML is a software tool used to create UML diagrams.\\
SM User: Social media user; the user who is logged into the system.\\
MVC: Model View Controller is a software architectural pattern commonly used for developing user interfaces which divides the related program logic into three interconnected elements.\\
Client/Server: Client–server model is a distributed application structure that partitions tasks or workloads between the providers of a resource or service, called servers, and service requesters, called clients.\\
Three Tier: Three-tier architecture is a client-server software architecture pattern in which the user interface (presentation), functional process logic ("business rules"), computer data storage and data access are developed and maintained as independent modules, most often on separate platforms.\\
\chapter{Approval page with signatures of the team Appendix} 
\chapter{References}
\chapter{Appendices}
\section{Appendix A}
\section{Appendix B}
\section{Appendix C}
\section{Appendix D}
\section{Appendix E}

\end{document}

\documentclass
