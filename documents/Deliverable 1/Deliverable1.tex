\documentclass[letter,12pt]{report}

\begin{document}


\title{Software Engineering\\CEN 4010-U01\\Blue Jay\\Team 2\\Deliverable 1}
\author{David Ramirez\and Josselyn Ruiz\and Alex Collantes\and Mareo Yapp\and Brandon Lee}
\date{\today}
\maketitle

\tableofcontents

\begin{abstract}
\addcontentsline{toc}{chapter}{Abstract}
This document describes the basic structure of the Blue Jay project.
The primary goal is to clearly describe the design of the project,
and begin laying out the groundwork for how the system will be laid out
in the future.
\end{abstract}



\chapter{Introduction}
This chapter covers the basic knowledge needed to understand our project.

\section{Purpose of the System}
The Blue Jay system is a social media application,
remniniscent of twitter and reddit. It shares some of its
core functionality withe such systems, namely the ability for users
to create posts, reply to posts, browse posts, and view posts.


\section{Scope of the System}
The system is meant to be a simplified version of a social media platform.
Users can create posts, which will be publicly viewable by all other users.
Users can view posts, and from viewing posts reply to posts with their own
posts.
Users can access posts through browsing through recent posts, searching
through posts, or from the home page.



\section{Development Methodology}
This software is being developed under the Unified Software Development Process.
As part of the 


\section{Definitions, Acronyms, Abbreviations}




\section{Overview of Document}
This document is divided into several chapters.
Chapter 2, Current Systems, covers existing systems and how our system
is different from those.
Chapter 3, Project Plan, covers our current plan for the project,
how we plan to organize the project, and what we tools we will need for the project.
Chapter 4, Requirements Elicitation, covers possible use cases for the project,
and the ten use cases we plan to implement.
Chapter 5, Requirements Analysis, covers

Chapter 6, The Glossary, defines terms used in the document.
Chapter 7, Approvals,
Chapter 8, References,


\chapter{Current System}



\chapter{Project Plan}

\section{Project Organization}

\section{Hardware and Software Requirements}

This section covers, what tools will be necessary
Not many special tools will be needed for running this project.
The following software is necessary for this project:
\begin{description}
\item [MySQL or PostgreSQL] Necessary for storing posts and user data,
precise version still undetermined at this point.
\item [Apache Server v2.4] Necessary for running of the site itself,
providing the ability to generate content through php scripts.
\item [Googe Chrome 77.0.+] Used for accessing website content during testing.
\item [Ubuntu 18.04] Used as host operating system for Apache and SQL server.
\end{description}

Hardware may or may not be outsourced to a cloud solution in the future,
but the minimum specs we expect to need are:

\begin{itemize}
\item Dual Core 2GHz or above processor.
\item 4 gibibytes of ram.
\item Ethernet connection to the internet, with speeds at or above 300 mbps.
\end{itemize}


\section{Project Schedule}






\chapter{Requirements Elicitation}

\section{Use Cases}
The following section covers the varirous possible use cases
that could be implemented for the system.

\subsection{T3-BJ001-Browse Posts}
\textbf{Use Case ID:} T3-BJ001-Browse Posts\\
\textbf{Use Case Level:} High-level\\
\textbf{Details:}
\begin{itemize}
    \item The homepage is to display all recently popular posts
    \item \textbf{Actor:} SM User
    \item \textbf{Pre-conditions:}
    \begin{itemize}
        \item User has internet access
        \item User has logged on.
    \end{itemize}
    \item \textbf{Description:}
    \begin{enumerate}
        \item User requests homepage
        \item System queries database for most popular posts within the last 24 hours
        \item Server constructs home page, showing most popular posts.
        \item Page is delivered to user browser.
        \item Browser displays homepage
    \end{enumerate}
    \item \textbf{Post-conditions:}
    \begin{itemize}
        \item Opening this page has no effect on the internal state of the server. The page is delivered to the client user/administrator.
    \end{itemize}
\end{itemize}
\textbf{Alternative Courses of Action:}
\begin{itemize}
    \item User searches for specific posts
    \item User checks his/her notifications
\end{itemize}
\textbf{Extensions:}
\begin{itemize}
    \item None
\end{itemize}
\textbf{Exceptions:}
\begin{itemize}
    \item Posts could not be retrieved.
\end{itemize}
\textbf{Concurrent Uses:}
\begin{itemize}
    \item None
\end{itemize}
\textbf{Related Use Cases:}
\begin{itemize}
    \item T3-BJ002-View Post
    \item T3-BJ003-View Profile
\end{itemize}
\textbf{Decision Support}
\begin{itemize}
    \item Frequency: 100 times per minute
    \item Criticality: High
    \item Risk: Medium
\end{itemize}
\textbf{Constraints:}
\begin{itemize}
    \item Usability: Average of about a minute for users to recognize and understand interface.
    \item Reliability: 99\% available.
    \item Performance: Page should load in less than a second.
    \item Supportability: Supported by browsers across multiple desktop and mobile devices.
\end{itemize}
\textbf{Modification History}
\begin{itemize}
    \item Owner: David Ramirez
    \item Initiation date: 8/27/2019
    \item Date last modified: 9/30/2019
\end{itemize}
\subsection{T3-BJ002-View Post}
\textbf{Use Case ID:} T3-BJ002-View Post\\
\textbf{Use Case Level:} High-level\\
\textbf{Details:}
\begin{itemize}
    \item %PUT DESCRIPTION HERE
    \item \textbf{Actor:} User
    \item \textbf{Pre-conditions:}
    \begin{itemize}
        \item User has internet access
        \item User has logged on.
    \end{itemize}
    \item \textbf{Description:}
    \begin{enumerate}
        \item User selects a post on the page while searching through the posts, or within any of the extensions of that functionality.
        \item The browser requests the page with the selected post.
        \item The post, and all posts replying to that post are queried from the database.
        \item The server constructs the page to display the posts.
        \item The server sends the page to the browser.
        \item The browser displays the page to the user.
    \end{enumerate}
    \item \textbf{Post-conditions:}
    \begin{itemize}
        \item Number of views on the post increases by one.
    \end{itemize}
\end{itemize}
\textbf{Alternative Courses of Action:}
\begin{itemize}
    \item N/A
\end{itemize}
\textbf{Extensions:}
\begin{itemize}
    \item N/A
\end{itemize}
\textbf{Exceptions:}
\begin{itemize}
    \item Post could not be retrieved.
\end{itemize}
\textbf{Concurrent Uses:}
\begin{itemize}
    \item None
\end{itemize}
\textbf{Related Use Cases:}
\begin{itemize}
    \item T3-BJ001-Browse Post
\end{itemize}
\textbf{Decision Support}
\begin{itemize}
    \item Frequency: 100 times per minute
    \item Criticality: High
    \item Risk: Medium
\end{itemize}
\textbf{Constraints:}
\begin{itemize}
    \item Usability: Average of 2 minutes for user to learn how to navigate pages and make posts
    \item Reliability:
    \item Performance:
    \item Supportability: Supported by browsers across multiple desktop and mobile devices.
    \item Implementation:
\end{itemize}
\textbf{Modification History}
\begin{itemize}
    \item Owner: David Ramirez
    \item Initiation date: 8/27/2019
    \item Date last modified: 9/30/2019
\end{itemize}






\section{Use Case Diagram}

\chapter{Requirements Analysis}

\section{Scenarios}

\section{Static Model}

\section{Dynamic Model}

\chapter{Glossary}

\chapter{Approvals}

\chapter{References}

\appendix
\chapter{Project Schedule}
\chapter{User Interface Designs}
\chapter{Meetings Diary}

\end{document}
